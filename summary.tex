\chapter{Summary} % (fold)
\label{cha:summary}

This thesis covers theoretical approaches to understand ultrafast processes in photoionization, Rydberg state excitation, and electron correlation.  In the first two chapters, an introduction to ultrafast and strong-field physics was given with a review of relevant experimental and theoretical studies. The numerical  and analytical methods that are used in this thesis were covered in Chap.~\ref{cha:numberic_methods}, including original studies using RBFs to solve the time independent Schr\"odinger equation (Sec.~\ref{sub:radial_basis_functions}) and a frequency correction to ultrashort laser pulses (Sec.~\ref{sec:laser_pulses}).

The focus was then turned to developing a method of imaging ultrafast electron wavepacket dynamics using few photon ionization in Chap.~\ref{cha:imaging_wave_packets_with_photoelectrons}. The extreme bandwidth of ultrafast pulses allows for processes involving different numbers of photons from the same pulse to interfere. The impacts on ionization of a helium atom were studied in Sec.~\ref{sec:short_pulse_effect} including the effects of pulse variations present in experiment. Sec.~\ref{sec:generalized_asymmetry_parameters} extends the typical forward backward ionization asymmetry from ground states to the ionization of wavepackets in superposition with different $m$ quantum numbers. To this end, we presented a novel set of generalized asymmetry parameters (GAP) to study the interference of the ground and excited state ionization signals. Finally, a method for reconstructing wavepackets undergoing attosecond electron motion using perturbation theory was presented in Sec.~\ref{sec:wavefunction_reconstruction}. The impact of shot to shot variations and the laser parameter regime required for a quality reconstruction have been analyzed.

Next, Chap.~\ref{cha:rydberg_state_excitations} covers the excitation of Rydberg states by intense IR laser pulses. Linear polarization results, in Sec.~\ref{sec:linear_polarization}, show that the $\ell$ quantum number of the populated states has the same parity as the sum of the number of photons involved in the process and the ground states $\ell$ quantum number. The FID signals present in the HHG spectrum allows for experimental verification of this effect. Sec.~\ref{sec:bi_circular_polarization} extended this study to bi-circular driving laser pulses, including an analysis of the pathways to available resonances and the occurence of high order Raman transitions which have been further studied in a project led by another graduate student \cite{gebre2021}.

Finally, we turned to the development and first tests of a fully correlated two electron code in Chap.~\ref{cha:electron_correlation}. We used hyperspherical harmonics to reduce the number of radii that must be discretized from two to one. The code produces ground and excited state energies consistent with experimental results and has produced a Rabi flop between the $1s$ and the $2p$ state in He that matches our SAE codes. Future development may allow for fully correlated two electron calculations to be performed efficiently on desktop computers opening the door for more in depth studies of time dependent multi-electron effects in ultrafast physics. 

% chapter summary (end)